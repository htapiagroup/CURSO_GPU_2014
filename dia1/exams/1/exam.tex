\documentclass{article}
\usepackage[latin1]{inputenc}
\usepackage{graphicx}
\begin{document}
\title{Escuela de Verano GPU2014}
\author{Examen 1}
\maketitle 
\parindent=0mm 
\begin{enumerate}
\item A continuaci�n se presenta un programa, que se puede compilar. � Qu� arroja cuando 
se ejecuta ?
\begin{enumerate}
\item Result = 20
\item Result = 14
\item Result = 28
\item Result = 2
\item Result = 4
\item Segmentation fault
\end{enumerate}
\includegraphics{mist.png}

\item A continuaci�n se presentan 3 breves programas (por orden de aparici�n {\tt test1.c}, {\tt test2.c} y {\tt test3.c}) que pretenden hacer lo siguiente:
  \begin{enumerate}
  \item Crear tres arreglos (vectores) de 10 elementos cada uno, llamados {\tt x}, {\tt y} y {\tt z}.
  \item Inicializar los valores de {\tt x} e {\tt y}.
  \item Multiplicar estos dos vectores elemento por elemento y poner los 10 resultados en el vector {\tt z}.
  \item Escribir los resultados en pantalla.
  \end{enumerate}
Contestar las siguientes preguntas:
\begin{enumerate}
\item � Cual de estos 3 programas va a imprimir los resultados ?
\item � Cual va a arrojar, ademas de los resultados, un mensaje de error ?
\item � Cual no va a arrojar ning�n resultado (y a lo mejor no va a compilar) ?
\item � Que mejor�a propondr�a al programa que arroja los resultados sin mensaje de error ?
\end{enumerate}
\includegraphics{test1.png}
\includegraphics{test2.png}
\includegraphics{test3.png}
\end{enumerate}
\end{document}